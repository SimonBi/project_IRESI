\documentclass[a4paper]{article}%

\usepackage[english]{babel}%
\usepackage[utf8]{inputenc}%
\usepackage[T1]{fontenc} %

\usepackage{graphicx}%
\usepackage{xspace}%

\usepackage{amsmath}
\usepackage{amsfonts}
\usepackage{amssymb}

\usepackage[margin=2.5cm]{geometry}

\usepackage{caption}
\usepackage{subcaption}
\usepackage{float}  % H option for figures


\begin{document}

\title{Project IRESI}

\author{Simon Bihel, Florestan De Moor \\ ENS Rennes, Computer Science Department, 1st year}

\date{November 22, 2015}

\maketitle

\begin{abstract}
	In this report we present our work on the project of the course IRESI. We will first explain how we implemented the sketch codeviance, which is a metric used to detect DDoS. We will then compare the results of the algorithm to exact values.
\end{abstract}


\section{DDoS and server security}
% Simon / Florestan
What's DDoS, difficulty of detecting it, need of distributed algorithm.


\section{Programming the SketchMin algorithm}
% mini intro 2l
For the detection we need to first select the entries to then apply the algorithm that will produce the material in which we will find whether or not there is an attack.

\subsection{Extracting information from data traces}
% Simon
Translate if needed to integers. ??? Could be repetitive with subsection Real data traces.

\subsection{Computing the codeviance}
% Florestan

\section{Experimental results}
% mini intro 2l
To evaluate the correctness of the algorithm, we just need to see if the results have the same appearance of the real values, with possibly a larger scale.

\subsection{Real data traces}
% Simon
Extracting all at once to maybe find correlation, even after applying the bijection. Extracting the file-names cause the third kind of traces was unusable if we wanted to analyze the sources, as it only says if an internal or external request.

\subsection{Generated data traces}
% Florestan


\section*{Conclusion}
% Simon / Florestan

\clearpage

\end{document}